\documentclass[11pt]{spidercv}
\usepackage[utf8x]{inputenc}
\usepackage[T1]{fontenc}
\usepackage[russian]{babel}
\usepackage{fontawesome}

\begin{document}

    %%%%%%% Printable mode
    \PrintableMode

    %----------------------------------------------------------------------------------------
    %	TOP BAR
    %----------------------------------------------------------------------------------------
    \begin{TopBar}{\ColorTextSide}
        \Name{\ColorHighlight}{Петров Олег}{}{}

        \TextSeparator{\ColorHighlight}{Желаемая должность}

        \begin{DoubleColumns}
            \begin{ItemList}{\ColorHighlight}
                    \item Intern/Junior Data Scientist
            \end{ItemList}
            \nextcolumn
            \begin{ItemList}{\ColorHighlight}
                    \item Intern/Junior Data Analyst
            \end{ItemList}
        \end{DoubleColumns}

        \vspace*{0.3cm}

        %%% Contact
        \TextSeparator{\ColorHighlight}{Контакты}
        % \SimpleSeparator{\ColorHighlight}
        \begin{DoubleColumns}
            \begin{ItemList}{\ColorHighlight}
                \item [\Large\faEnvelopeO] \href{mailto:oepetrov0@gmail.com}{oepetrov0@gmail.com}
                \item [\Large\faPaperPlaneO] \href{https://t.me/pauchara0}{pauchara0}
            \end{ItemList}
            \nextcolumn
            \begin{ItemList}{\ColorHighlight}
                \item [\Large\faGithub] \href{https://github.com/necroshine0}{necroshine0}
                % \item [\Large\faLinkedinSquare] \href{https://www.linkedin.com/}{romainpc}
            \end{ItemList}
            \nextcolumn
            \begin{ItemList}{\ColorHighlight}
                % \item [\Large\faLinkedinSquare] \href{https://www.linkedin.com/}{romainpc}
            \end{ItemList}
        \end{DoubleColumns}

    \end{TopBar}



    %----------------------------------------------------------------------------------------
    %	SIDE BAR
    %----------------------------------------------------------------------------------------
    \begin{SideBar}{\ColorBackground}{\ColorTextSide}
        
        %%% Hard skills
        \TextSeparatorBis{\ColorHighlight}{Hard skills}{\faGears}
        \begin{ItemList}{\ColorHighlight}
            \item [\faCode]Code: Python, C++
            \item [\faDatabase]Data: numpy, pandas, SQL
            \item [\faBarChartO]Draw: matplotlib, seaborn, shap
            \item [\faSpinner]ML: scikit-learn, catboost,\\ \ \ \ \ \ statsmodels, etc.
            \item [\faConnectdevelop]DL: pytorch, gensim,\\ \ \ \ \ \ transformers, gluonts
            \item [\faLanguage]Английский язык (B2)
        \end{ItemList}

        \vspace*{0.4cm}
        
        %%% Soft skills
        \TextSeparatorBis{\ColorHighlight}{Soft skills}{\faUser}

        \Label{\ColorHighlight}{Обучаемость}
        \Label{\ColorHighlight}{Педантичность}
        \Label{\ColorHighlight}{Адаптивность}
        \Label{\ColorHighlight}{Коммуникативность}

        \vspace*{0.7cm}

        \TextSeparatorBis{\ColorHighlight}{Skills graph}{\faCodepen}

        \begin{SpiderDiagram}{\ColorTextSide}{\ColorHighlight}
            \addSkill{ML}{5}
            \addSkill{DL}{3}
            \addSkill{Math}{4}
            \addSkill{Analysis}{5}
            \addSkill{DevOps}{2}
            
        \end{SpiderDiagram}

        \vspace*{0.9cm}

        %%% Languages
        \TextSeparatorBis{\ColorHighlight}{Прочее}{\faObjectGroup}
            
        \begin{SkillGauges}{\ColorHighlight}
            \addGauge{Yandex Datasphere}{4}
            \addGauge{Yandex Toloka}{3}
            \addGauge{Bash}{3}
            \addGauge{Git}{2}
        \end{SkillGauges}
        
    \end{SideBar}

    

    %----------------------------------------------------------------------------------------
    %	Picture
    %----------------------------------------------------------------------------------------
    \DefineProfile{\ColorOther}{\ColorTextSide}{img/cv photo.jpg}




    %----------------------------------------------------------------------------------------
    %	Main
    %----------------------------------------------------------------------------------------
    \begin{MainPart}

    %%% Education
    \MainTitle{\ColorHighlight}{\ColorTextMain}{Образование}
    \Experience
        {\ColorHighlight}
		{\Large{Бакалавр прикладной математики и информатики}}
		{НИУ ВШЭ}
        {2019-2023}
        {   
            Специализация <<Машинное обучение и приложения>>, ФКН.\\
            % Тема диплома: <<Поиск новой физики в данных с помощью генеративных моделей>>.\\
        }

    %%% Experiences 
    \MainTitle{\ColorHighlight}{\ColorTextMain}{Опыт работы}
    \Experience
        {\ColorHighlight}
		{\Large{Учебный ассистент}}
		{НИУ ВШЭ}
            {С сентября 2021-\faUndo}
        {   
            Ассистировал курсы по высшей математике, разработке на Python, теории вероятностей и машинному обучению.\\

            Обязанности: проверка работ студентов, консультирование, помощь в организации учебного процесса.
        }

    \MainTitle{\ColorHighlight}{\ColorTextMain}{Проекты}
    \Experience
        {\ColorHighlight}
		{\Large\faGithub\ \href{https://github.com/necroshine0/edu-analytics}{Образовательная аналитика}}
            {}{}
        {   
            Цель: факторизация учебных планов образовательных программ на основе кластеризации документов аннотаций учебных дисциплин.\\
            
            Сбор данных из открытых источников, валидация и разметка; кластеризация на векторных представлениях документов; метрики Xie-Beni и B-Cubed для оценки качества кластеризации.\\
        }

    \Experience
        {\ColorHighlight}
		{\Large\faGithub\ \href{https://github.com/necroshine0/ds-notebooks}{Data Science and Kaggle repo}}
            {}{}
        {   
            Данный репозиторий показывает мои навыки самостоятельного решения задач машинного обучения и аналитики данных.\\
        }

    \Experience
        {\ColorHighlight}
		{\Large\faGithub\ \href{https://github.com/necroshine0/database-theory-proj}{Database Telegram Bot}}
            {}{}
        {   
            Разработал Telegram-бота, выполняющего get-запросы к базе данных (SQLite3). Участвовал в разработке концептуальной модели базы данных и DDL, полностью прописал DML. Высшая оценка за проект на курсе.\\
        }

    \Experience
        {\ColorHighlight}
		{\Large\faGithub\ \href{https://github.com/necroshine0/maze-proj}{Дискретные операторы}}
            {}{}
        {   
            Программно доказал возможность моделирования поведения мыши в лабиринте с помощью дискретных операторов. Научный консультант: В.В. Толченникова, Биологический факультет МГУ. \\
        }

    \Experience
        {\ColorHighlight}
		{\Large\faGithub\ \href{https://github.com/kaledinandrew/TicTacToeOnline}{Tic-Tac-Toe Online}}
            {}{}
        {   
            Разработал product vision, user stories и UML. Занимался разработкой клиентского приложения на Java (frontend).
        }

    \end{MainPart}

    

\end{document}


